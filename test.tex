\documentclass{gunote}
% preamble
\usepackage{amsmath, tabularray, hyperref}
\newminted{latex}{rulecolor=mahiropink2}
\newminted{cpp}{rulecolor=mahirodark}
\newminted{python}{rulecolor=asahigreen}
\newminted{text}{rulecolor=asahibrown}
\title{gunote \LaTeX{}笔记模板}
\author{Gemini Usagi,School of Geodesy and Geodetic,Wuhan Univeristy}
\date{\today}

\newcommand{\cmd}[1]{\texttt{\backslash #1}}
\def\mbi{\symbfit}
\def\mbu{\symbf}
% main
\begin{document}
\maketitle
\tableofcontents

\section{字体测试}
\textsf{gunote}模板为自用模板,基于\textsf{fontspec}宏包、\textsf{unicode-math}宏包和\textsf{ctex}宏包进行了自定义的字体设置。
\subsection{英文字体}
英文字体设置,罗马字族设置为TeX Gyre Pagella,该字体大多数\TeX{}Live发行版的用户应该都有,在命令行中输入
\begin{minted}{text}
fc-match -v 'TeX Gyre Pagella'
\end{minted}
可以查看自己是否拥有字体,以及字体所在的路径。
\begin{minted}{latex}
\setmainfont{TeX Gyre Pagella}
\end{minted}
由于TeX Gyre Pagella字体具有\texttt{-regular}、\texttt{-bold}、\texttt{-italic}和\texttt{-bolditalic}的设计,因此使用命令
\begin{minted}{latex}
\textbf{some text}
\textit{some text}
{\bfseries\itshape some text}
\end{minted}
可以分别得到如下的效果:\textbf{some text}\quad\textit{some text}\quad{\bfseries\itshape some text}.

无衬线字族设置为\textsf{Gill Sans MT},打字机字族设置为\texttt{JetBrains Mono},前者为Windows平台下的默认字体,后者为JetBrains公司开发的免费开源字体\footnote{下载网址:\url{https://www.jetbrains.com/zh-cn/lp/mono/}}。
\subsection{数学字体}
通过\textsf{unicode-math}宏包提供的\cmd{setmathfont}命令可以很方便地设置数学字体,本模板使用的数学字体为TeX Gyre Pagella Math,效果如下:
{
\def\rcv{\mathrm{r}}
\def\sat{\mathrm{s}}
\begin{gather}
  p_{\rcv,j}^\sat=\rho_\rcv^\sat+c(dt_\rcv-dt^\sat)+T_\rcv^\sat+I_{\rcv,j}^\sat+e_{\rcv,j}^\sat,\\
  \varphi_{\rcv,j}^\sat=\rho_\rcv^\sat+c(dt_\rcv-dt^\sat)+T_\rcv^\sat-I_{\rcv,j}^\sat+\lambda_j N_{\rcv,j}^\sat+\varepsilon_{\rcv,j}^\sat.
\end{gather}
}

在矩阵和向量的表示上,作者所在的专业往往采用加粗的方式。得益于\textsf{unicode-math}宏包对数学字体的处理机制,推荐使用\cmd{symbfit}命令得到粗斜数学字体。
\begin{gather}
  \hat{\mbi{x}}_{k\mid k-1}=\mbi{\Phi}_{k\mid k-1}\hat{\mbi{x}}_{k-1},\\
  \hat{\mbi{P}}_{k\mid k-1}=\mbi{\Phi}_{k\mid k-1}\hat{\mbi{P}}_{k-1}\mbi{\Phi}_{k\mid k-1}^\top+\mbi{Q}_k.
\end{gather}
有时习惯采用直立而非倾斜的数学字体,推荐使用\cmd{symbf}命令
\begin{gather}
  \hat{\mbu{x}}_{k\mid k-1}=\mbu{\Phi}_{k\mid k-1}\hat{\mbu{x}}_{k-1},\\
  \hat{\mbu{P}}_{k\mid k-1}=\mbu{\Phi}_{k\mid k-1}\hat{\mbu{P}}_{k-1}\mbu{\Phi}_{k\mid k-1}^\top+\mbu{Q}_k.
\end{gather}
\subsection{中文字体}
中文字体通过\textsf{ctex}宏包对\textsf{xeCJK}宏包内容的调用,以及预定义的一些命令,实现自定义中文字体。通过输入
\begin{minted}{latex}
\LoadClass[fontset=none]{ctexart}
\end{minted}
实现对\textsf{ctex}宏包的调用,同时声明自定义字体。中文默认宋体字族为思源宋体Source Han Serif CN,黑体字族为苹方字体.PingFang SC,楷体字族为系统默认楷体KaiTi.

文中的很多场合需要用到加粗的文字来表示强调,此时推荐使用
\begin{minted}{latex}
\textbf{强调文字}
\end{minted}
来实现,效果:\textbf{强调文字}。此时使用的是思源宋体的Bold样式。

如果习惯于Word那样黑体加粗的格式,可以使用
\begin{minted}{latex}
{\bfseries\sffamily 黑体加粗}
% 或者
\textbf{\heiti 黑体加粗}
\end{minted}
来实现,效果:{\bfseries\sffamily 黑体加粗}。此时使用的是苹方字体的Medium样式。
\section{颜色测试}
作者基于\textsf{xcolor}宏包,参考Onimai Character\footnote{\url{https://onimai.jp/character/}}设计了部分颜色,并定义了一些颜色命令,如果你希望在文档中使用这些颜色,请参照表\ref{tab:color}:
\begin{table}[htbp]
  \centering
  \caption{\label{tab:color} 预定义颜色}
  \begin{tblr}{
    colspec={cccc|cccc},
    cell{2}{1}={r=5}{c},
    cell{2}{3}={mahirolight},
    cell{3}{3}={mahirodark},
    cell{4}{3}={mahiropink1},
    cell{5}{3}={mahiropink2},
    cell{6}{3}={mahirogray},
    cell{3}{5}={r=3}{c},
    cell{3}{7}={miharigold},
    cell{4}{7}={miharipurple},
    cell{5}{7}={miharired},
    cell{7}{1}={r=2}{c},
    cell{7}{3}={momijibrown},
    cell{8}{3}={momijiblue},
    cell{7}{5}={r=2}{c},
    cell{7}{7}={kaedepink},
    cell{8}{7}={kaedeblue},
    cell{9}{1}={r=2}{c},
    cell{9}{3}={asahibrown},
    cell{10}{3}={asahigreen},
    cell{9}{5}={r=2}{c},
    cell{9}{7}={miyopurple},
    cell{10}{7}={miyoyellow}
  }
    \hline
    取材角色 & 名称 & 预览 & RGB值 & 取材角色 & 名称 & 预览 & RGB值 \\
    \hline
    緒山まひろ & mahirolight &  & 228,243,248 &  &  &  &  \\
     & mahirogray &  & 133,149,174 & 緒山みはり & miharigold &  & 245,195,134 \\
     & mahiropink1 &  & 234,157,169 &  & miharipurple &  & 221,157,240 \\
     & mahiropink2 &  & 234,212,207 &  & miharired &  & 234,140,156 \\
     & mahirodark &  & 125,134,156 &  &  &  &  \\
    \hline 
    穂月もみじ & momijibrown &  & 151,119,128 & 穂月かえで & kaedepink &  & 241,172,184 \\
     & momijiblue &  & 152,213,238 &  & kaedeblue &  & 103,210,231 \\
    \hline
    桜花あさひ & asahibrown &  & 188,153,134 & 室崎みよ & miyopurple &  & 185,133,145 \\
     & asahigreen &  & 119,147,109 &  & miyoyellow &  & 254,234,153 \\
    \hline
  \end{tblr}
\end{table}
部分颜色被有机地穿插进了文档的各个元素中。
\section{自定义环境测试}
\subsection{代码环境}
\textsf{gunote}模板对代码环境进行了简单的设定,基于\textsf{minted}宏包。效果如下:
\begin{minted}[highlightlines=5]{latex}
\doucumentclass{gunote} % gunote template
% preamble
\begin{document}
% main
  \textcolor{mahirogray}{some text} % highlight here
\end{document}
\end{minted}
\end{document}